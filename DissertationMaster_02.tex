% \documentclass[nohyperref]{UMassThesis09}
\documentclass[nohyperref,altsigspacing]{UMassThesis09}
\usepackage{microtype}

% the microtype package allows 'tracking', which is uniform addition or subtraction) of space between all the characters in text. We will apply tracking to text that is in all caps, such as the title and the author name on the title page. This kind of tracking is done using microtype's command '\textls'.


% The '\providecommand{\textls}[2][]{#2}' below is there in case you decide not to use microtype. Note that '\textls' must be defined one way or another, because it is used in the class file. If the microtype package is used, then it will define the command '\textls', and the '\providecommand' will be ignored. On the other hand, if the microtype package is not used, then the '\providecommand' defines '\textls' in such a way that '\textls' does nothing. 

\providecommand{\textls}[2][]{#2}

% Here we define the tracking amounts for the all-caps title and the all-caps author name. Of course, if microtype wasn't used, these will actually do nothing.
\newcommand{\titlels}[1]{\textls[75]{#1}} 
\newcommand{\authorls}[1]{\textls[50]{#1}} 

\setcounter{tocdepth}{2}

\usepackage{physics}
\usepackage{verbatim,listings}
\usepackage{graphicx,xcolor,enumitem}
\usepackage{cite}
\usepackage{bm}
\usepackage{pdfpages}

\graphicspath{ {Images/} }

\usepackage{epstopdf}
\usepackage{float}
%Document

\newcommand{\arxiv}[2][]{\ifthenelse{\isempty{#1}}{\href{http://arxiv.org/abs/#2}{{\tt arXiv:\allowbreak{}#2}}} {\href{http://arxiv.org/abs/#2}{{\tt arXiv:\allowbreak{}#2 [#1]}}}}
\newcommand{\pirsa}[1]{\href{http://pirsa.org/#1/}{{\tt PIRSA:\allowbreak{}#1}}}
\newcommand{\booktitle}{\textsl}
\newcommand{\hrefdoi}[2]{\href{https://dx.doi.org/#1}{#2}}

\newcommand\invisiblesection[1]{%
  \refstepcounter{section}%
  \addcontentsline{toc}{section}{\protect\numberline{\thesection}#1}%
  \sectionmark{#1}}

\newtheorem{theorem}{Theorem}[section]
\newtheorem{corollary}{Corollary}[theorem]
\newtheorem{lemma}[theorem]{Lemma}
\newtheorem{definition}{Definition}[section]
%
 
%
\captionsetup[sub]{font=small,textfont=rm,labelformat=left-justify_label}        
%

\newcommand{\slfcr}{sc}
\newcommand{\xsc}[1]{x^{\text{(sc)}}_{#1}}


\usepackage{relsize}
\usepackage{xspace}
%\usepackage[per-mode=symbol,range-phrase = \text{--}]{siunitx}
% \usepackage{xr} % this is a package for cross-referencing external documents; see https://www.overleaf.com/learn/how-to/Cross_referencing_with_the_xr_package_in_Overleaf
\usepackage[cal=boondoxo]{mathalfa}
\usepackage[defaultlines=2,all]{nowidow}
\usepackage{layout}




\begin{document}
\doublespacing
\setlength{\parindent}{0.5in}

\newcommand{\TitleName}{New mathematical approaches to ultra-cold atoms}
\author{Joanna Ruhl} % don't use all capitals here
% \newcommand{\AuthorDegrees}{B.S., University of California, Los Angeles\\
% M.S., University of Massachusetts Boston\\
% Ph.D., University of Massachusetts Boston}


\AuthorDegrees{B.S., University of California, Los Angeles\\
M.S., University of Massachusetts Boston\\
Ph.D., University of Massachusetts Boston}

\DirectedBy{Professor Maxim Olchanyi} % normally this should be the same person as in the entry for \CommitteeMemberOne, below 
\ProgramName{Applied Physics Program} 
\DegreeMonthYear{May}{2025} % the month will be stored in \DegreeMonth , and the year in \DegreeYear

\ThesisOrDissertation{Dissertation}
\DegreeName{Doctor of philosophy} 

\ProgramDirector{%
Chandra Yelleswarapu, Program Director\\
Applied Physics Program
}

% \DeptChairperson{%
% Ricardo Castaño-Bernard, Interim Dean\\
% College of Science and Mathematics
% }

\DeptChairperson{%
Rahul Kulkarni, Chair\\
Department of Physics
}


%%% Normally, the number of committee members is three or four (plus  the program director and the department chair). 
%%% Indeed, most universities/departments have a formal minimum of three (e.g. general UMass Boston [https://tinyurl.com/yzrfjhac], general Northeastern [https://tinyurl.com/mr3hsh3x]) or four (e.g.  UMass Boston Physics, UPenn Computing [https://tinyurl.com/6p6kk29h]), though some have five (e.g. U. Wisc.–Madison Mechanical Engineering [https://tinyurl.com/454ejm5m]). 
%%% While most universitites/departments don't set an explicit upper limit on the number of committee members, some do. For example, some have formally set it at five (Stanford [https://tinyurl.com/24hc4rsb], U. South. Calif. [https://tinyurl.com/3ywhdpsx]), six (U. Ill. Chicago Physiology and Biophysics [https://tinyurl.com/mr29xn5y]), seven (UPenn Computing [https://tinyurl.com/6p6kk29h], Bowie State Computer Science [https://tinyurl.com/2s3xmu32]), or eight (U. Wisc.–Madison Mechanical Engineering [https://tinyurl.com/454ejm5m]).

%%% In principle, the template and the class file can accept an arbitrary number of committee members without reporting an error. 
%%% (OK, CS people, don't take this too literally. Yes, you'll get numerical overflows eventually---with default settings, at 16382 members.)

%%% However, if there are NINE or MORE committee members (plus the program director and the department chair), then this class file and template will be unlikely to produce acceptable results, and some modification of them will be required. For example, one can perhaps introduce multiple signature pages, or reduce the font size of the committee member info (as well as that of the program director and the department chair). Before you do anything, however, you really should contact the Office of Graduate Studies (graduate.studies@umb.edu) and ask them what they recommend you do.

%%% If you have FIVE or MORE committee members (but, as we said, no more than eight),
%%% then it is recommended that you move the topmost and bottommost signature 
%%% lines further apart. This is done by setting \TopRefSigLineVOffset and
%%% \BottomRefSigLineVOffset . What follows are recommended values for them.

%%% Additionally, you should consider loading the class file with the option 'altsigspacing'. If you have SIX or MORE committee members, using this option may be necessary to get acceptable results.

%% If you have FIVE committee members, here are 
%%% the recommended offsets (i.e., uncomment the next two lines)
% \setlength{\TopRefSigLineVOffset}{-25pt}
% \setlength{\BottomRefSigLineVOffset}{25pt}

%% if you have SIX, SEVEN, or EIGHT committee members, 
\setlength{\TopRefSigLineVOffset}{-50pt}
\setlength{\BottomRefSigLineVOffset}{50pt}

%% Additionally, if you have SIX or MORE committee members, you really should load the class file with the option 'altsigspacing'.


\AddCommitteeMember{%
Maxim Olchanyi, Professor\\
Chairperson of Committee
}

\AddCommitteeMember{%
Nathan Harshman, Professor\\
American University\\
Member
}

\AddCommitteeMember{%
Olga Goulko, Assistant Professor\\
Member
}



\AddCommitteeMember{%
Stephen Arnason, Associate Professor\\
Member
}

\AddCommitteeMember{%
Nathan Harshman, Professor\\
American University\\
Member
}
% 
% 
% \AddCommitteeMember{%
% Stephen Arnason, Associate Professor\\
% Member
% }

\AddCommitteeMember{%
Nathan Harshman, Professor\\
American University\\
Member
}




\AddCommitteeMember{%
Nathan Harshman, Professor\\
American University\\
Member
}


\pagenumbering{roman}
\setcounter{page}{1}
\pagestyle{plain}

\PrintTitlePage




\mbox{}\newpage
\thispagestyle{empty} 
\PrintCopyrightPage

\setlength{\SigPagePresentedByOffset}{6bp} % this is to (optionally) vertically move, on the signature page, the block that begins with 'A Dissertation/Thesis Presented by...' and ends with 'Approved as to style and content by:'. 

\mbox{}\newpage
\thispagestyle{empty}


% \PrintSignaturetPage

%%%%%%%%%%%%%%%%%%%%%%%%%%%%%%%%%%%%%%%%%%%%%%%%%%%%%%%%%%%%%%%%%%
%%%%%%%%%%%%%%%%%%%%%%%%%%%%%%%%%%%%%%%%%%%%%%%%%%%%%%%%%%%%%%%%%%
%%%%%%%%%%%%%%%%%%%%%%%%%%%%%%%%%%%%%%%%%%%%%%%%%%%%%%%%%%%%%%%%%%
\makeatletter
%
\newcommand{\@SignaturesSpacingInterp}{0}
\edef\textwidthInBp{\convertto{bp}{\textwidth}}
\begin{textblock}{\textwidthInBp}(\LeftMarginInBp,74.5)
\centering
% \begin{spacing}{1.1}
\begin{doublespace}
\titlels{\MakeUppercase{\TitleName}}
\end{doublespace}
% \end{spacing}
{%
% \vspace{15bp}
% \SigPagePresentedByOffset should be defined in the main .tex doucument. By default it's 0bp.
% \providelength{\SigPagePresentedByOffset}{0bp} % this is in case it hasn't been defined yet
\newlength{\TMPLengthA}\setlength{\TMPLengthA}{25bp+\SigPagePresentedByOffset}
\vspace{\TMPLengthA} 
% \vspace{25bp} 
\centering
\begin{spacing}{1.15}
 \setlength{\AuthorNameByVOffset}{\@AuthorNameByVOffset\baselineskip-\baselineskip}
 A\xspace \@ThesisOrDissertation\xspace Presented \\
 by\\
 \mbox{}\vskip\AuthorNameByVOffset\authorls{\MakeUppercase{\@author}} 
\end{spacing}
}
\vspace{18bp}
Approved as to style and content by:
\end{textblock}

\newcommand{\CommitteeChairLineVertDistRef}{\mymath 300.1 pt + \convertto{pt}{\TopRefSigLineVOffset} pt}
\newcommand{\DeptChairLineVertDistRef}{\mymath 646.6 pt + \convertto{pt}{\BottomRefSigLineVOffset} pt}




\ifAltSigSpacing
\begin{spacing}{\SignaturePersonSpacing}
%
% \makeatletter
\ifdefined\@ProgramDirector
\MeasureBox{\@ProgramDirector}
\setlength{\HeighProgDir}{\TotalHeightLength + \SignaturePersonLineToNameVertDist bp}
\global\HeighProgDir=\HeighProgDir
\fi
% \makeatother
%
% \ifdefined\@DeptChairperson
% \MeasureBox{\@DeptChairperson}
% \setlength{\HeighDeptChair}{\TotalHeightLength + \SignaturePersonLineToNameVertDist bp}
% \global\HeighDeptChair=\HeighDeptChair
% \fi
%
\end{spacing}
\fi

\ifAltSigSpacing
  \setlength{\CummulativeHeightOfBoxes}{\CummulativeHeightOfBoxes+\HeighProgDir}
\else
  \setlength{\CummulativeHeightOfBoxes}{0pt}
\fi

\newcommand{\ProgDirCount}{0}
\ifdefined\@ProgramDirector
\renewcommand{\ProgDirCount}{1}
\fi

\newcommand{\ChairCount}{0}
\ifdefined\@DeptChairperson
\renewcommand{\ChairCount}{1}
\fi

% \newcommand{\CommitteLinesVertDist}{\mymath(\DeptChairLineVertDistRef pt - \CommitteeChairLineVertDistRef pt - \convertto{pt}{\CummulativeHeightOfBoxes} pt)/(\mymath(\numCommMembr pt +\ProgDirCount pt + \ChairCount pt - 1 pt))}

% \numCommMembr=16381
\newcommand{\CommitteLinesVertDist}{\mymath(\DeptChairLineVertDistRef pt - \CommitteeChairLineVertDistRef pt - \convertto{pt}{\CummulativeHeightOfBoxes} pt)/(\mymath(\numCommMembr pt +\ProgDirCount pt + \ChairCount pt - 1 pt))}
% \mbox{}\vskip 1in \noindent a=\CommitteLinesVertDist\\
% \the\numCommMembr
% \numCommMembr=7\\
% \noindent b=\CommitteLinesVertDist



\newcommand{\SignaturePersonTextwidthInBp}{\textwidthInBp}
\newcommand{\SignaturePersonLeftMarginInBp}{\LeftMarginInBp}

\newcommand{\CurrentSignaturePersonVerticalPosition}{\mymath \CommitteeChairLineVertDistRef pt + \convertto{pt}{\MemberOneVOffset} pt}



% \mbox{}\vskip 0.5in
\newlength{\UpdaredCummulativeHeightBox}
\setlength{\UpdaredCummulativeHeightBox}{0pt}
\newcommand{\UpperLimit}{\mymath \the\numCommMembr pt + 1 pt}
\newcount\CurrentMember
\CurrentMember=1
\loop
  \ifnum\CurrentMember<\UpperLimit
   \SignaturePerson{\csname CommitteeMember\the\CurrentMember\endcsname}
   \ifAltSigSpacing
      \setlength{\UpdaredCummulativeHeightBox}{\UpdaredCummulativeHeightBox + \csname HeightBox\the\CurrentMember\endcsname pt}
   \fi   
   \edef\CurrentSignaturePersonVerticalPosition{\mymath \CommitteeChairLineVertDistRef pt + (\CommitteLinesVertDist pt)*\the\CurrentMember + \UpdaredCummulativeHeightBox + \MemberTwoVOffset}
   \advance \CurrentMember 1
\repeat
% 
% \numCommMembr=3
% \newcommand{\UpperLimit}{\mymath \the\numCommMembr pt + 1 pt}
% \newcount\CurrentMember
% \CurrentMember=1
% \loop
%   \ifnum\CurrentMember<\UpperLimit
%     \mbox{}\hfill\the\CurrentMember\\
%     \advance \CurrentMember 1
% \repeat


% \expandafter\ifx\csname CommitteeMember4\endcsname\relax
% Undefined
% \else
% Defined
% \fi
% %

% For the last two signature persons, shift the entries to the right
% We also change the length of thr signature line just a bit, to better match what's in Standards-UMB
\setlength{\SignaturePersonLineLength}{247.5pt}
\renewcommand{\SignaturePersonLeftMarginInBp}{288.0}



\ifdefined\@ProgramDirector
\renewcommand{\CurrentSignaturePersonVerticalPosition}{\mymath \CommitteeChairLineVertDistRef pt + (\CommitteLinesVertDist pt)*\numCommMembr + \UpdaredCummulativeHeightBox + \convertto{pt}{\ProgDirVOffset} pt}
\SignaturePerson{\@ProgramDirector}
\fi

\ifdefined\@DeptChairperson
\renewcommand{\CurrentSignaturePersonVerticalPosition}{\mymath \DeptChairLineVertDistRef pt + \convertto{pt}{\DeptChairVOffset} pt}
\SignaturePerson{\@DeptChairperson}
\fi
\makeatother

%%%%%%%%%%%%%%%%%%%%%%%%%%%%%%%%%%%%%%%%%%%%%%%%%%%%%%%%%%%%%%%%%%
%%%%%%%%%%%%%%%%%%%%%%%%%%%%%%%%%%%%%%%%%%%%%%%%%%%%%%%%%%%%%%%%%%
%%%%%%%%%%%%%%%%%%%%%%%%%%%%%%%%%%%%%%%%%%%%%%%%%%%%%%%%%%%%%%%%%%

\mbox{}\newpage
\setlength{\AbstractVerticalOffset}{0bp} % this is to vertically move, on the first page of the abstract, the material after the title. Suggested values: if the title is one line, 0bp; if the title is two lines, 36bp.

\begin{UMBAbstract}
 In this dissertation\ldots
\end{UMBAbstract}

\newpage
\iftestA
\topmatter{\testword}
\else
\topmatter{ACKNOWLEDGMENTS}
\fi
\doublespacing
Thanks go to\ldots

\newpage
\PrintToCLoFLoT


% ----------------------------------------------------
%     Chapters
% ----------------------------------------------------
\newpage
\pagestyle{plain}
\pagenumbering{arabic}
\setcounter{page}{1}
\normalsize

\UMBchapter{INTRODUCTION}

\singlespacing
\clearpage

\addtocontents{toc}{\protect\addvspace{10pt}}
\titlecontents{chapter}[0in]{\normalfont}{}{}{\titlerule*[3.3pt]{.}\contentspage[\thecontentspage\hbox{\hspace{\ToCPageNumToRightMargin}}]}
\addcontentsline{toc}{chapter}{LIST OF REFERENCES}
\bibliographystyle{ieeetr}
\bibliography{WKBRef}

\end{document}
